\documentclass[11pt]{beamer}
\usepackage[utf8]{inputenc}
\usepackage{xeCJK}
\usepackage[T1]{fontenc}
\usepackage{amsmath}
\usepackage{amsfonts}
\usepackage{amssymb}
%\usepackage[colorlinks, linkcolor=red]{hyperref}
%\usepackage{theorem}
\usepackage{graphicx}
%\usetheme{Hannover}
\usetheme{Berkeley}

\newtheorem{prob}{问题}[section]
\newtheorem{defi}{定义}[section]
\newtheorem{prop}{性质}[section]
\newtheorem{thrm}{定理}[section]
\newtheorem{exmp}{例}[section]
\newtheorem{rmk}{Remark}[section]

\def\mathfamilydefault{\rmdefault}
\def\dim{\mathrm{dim}}
\def\e{\mathrm{e}}

\begin{document}
	\author{王逸飞}
	\title{线性代数}
	\subtitle{物理学院学术辅导}
	\logo{\includegraphics[height=1.0cm]{logo.png}}
	\institute{北京大学物理学院}
	%\date{}
	%\subject{}
	%\setbeamercovered{transparent}
	%\setbeamertemplate{navigation symbols}{}
\begin{frame}[plain]
	\titlepage
	\scriptsize Click  {\color{red}\url{https://github.com/Florestan-Eusebius/Linear-Algebra}} to get the newest version.
\end{frame}

\frame{\tableofcontents}

\section{一些不严谨且没用的闲话}
\frame{\sectionpage}

\begin{frame}{线性关系}
	\begin{prob}
		我们见过的最简单数量关系是什么?\\
		我们见过的最简单的那些几何对象是什么?\\
		它们有怎样的性质和联系?
	\end{prob}
	\begin{prob}
		面对不那么简单的关系和对象, 我们怎样处理?\\
	\end{prob}
\end{frame}

\begin{frame}{线性关系}
	\begin{columns}
		\column{.5\textwidth}
		\begin{block}{任务}
			\begin{itemize}
				\item 研究线性关系.
				\item 研究更丰富的关系.
			\end{itemize}
		\end{block}
		\column{.5\textwidth}
		\begin{block}{手段}
			\begin{itemize}
				\item 代数方法$\rightarrow$线性代数.
				\item 分析方法$\rightarrow$微积分.
			\end{itemize}
		\end{block}
	\end{columns}
\end{frame}

\begin{frame}{线性关系}
\begin{columns}
	\column{.47\textwidth}
	\begin{block}{任务}
		\begin{itemize}
			\item 研究线性关系.
			\item 研究更丰富的关系.
		\end{itemize}
	\end{block}
	\column{.47\textwidth}
	\begin{block}{手段}
		\begin{itemize}
			\item 代数方法$\rightarrow$线性代数.
			\item 分析方法$\rightarrow$微积分.
		\end{itemize}
	\end{block}
\end{columns}
\begin{prob}
	什么是关系?\\
	什么是代数?
\end{prob}
\end{frame}

\begin{frame}{请站稳扶好, 注意安全}
好了, 废话说完了, 让我们进入抽象的世界.
\end{frame}

\section{线性空间}

\frame{\sectionpage}

\subsection{准备概念}

\frame{\subsectionpage}

\begin{frame}{复数}
	\begin{defi}[复数]
		\begin{enumerate}
			\item 复数, 复数集$\mathbb{C}$
			\item 加法和乘法
		\end{enumerate}
	\end{defi}
	\begin{prop}[复数]
		\begin{enumerate}
			\item 加法交换律
			\item 加法结合律
			\item 单位元(加法单位元0和乘法单位元1)
			\item 加法逆元
			\item 乘法逆元
			\item 分配律
		\end{enumerate}
	\end{prop}
\end{frame}

\begin{frame}{$\mathbb{F}^n$}
	\begin{defi}[list]
		$\left(x_1,\cdots,x_n\right).$
	\end{defi}
	\begin{defi}[$\mathbb{F}^n$]
		$\mathbb{F}^n=\left\{\left(x_1,\cdots,x_n\right):x_j\in\mathbb{F}\ \forall j=1,\cdots,n \right\}. $
	\end{defi}
	\begin{rmk}
		$\mathbb{F}$是所谓"数域", 指$\mathbb{C}$或$\mathbb{R}$. 我们暂不讨论其他域.
	\end{rmk}
\end{frame}

\begin{frame}{$\mathbb{F}^n$上的运算}
	\begin{defi}[加法]
		$\left(x_1,\cdots,x_n\right)+\left(y_1,\cdots,y_n\right)=\left(x_1+y_1,\cdots,x_n+y_n\right).$
	\end{defi}
	\begin{prop}[加法交换律]
	\end{prop}
	\begin{defi}[0]
	\end{defi}
	\begin{defi}[加法逆元]
	\end{defi}
	\begin{defi}[数乘]
	\end{defi}
\end{frame}

\begin{frame}{域(补充)}
	\begin{defi}[环]
		定义了加法和乘法运算的非空集合被称为环当且仅当这两个运算满足
		\begin{enumerate}
			\item 加法结合律
			\item 加法交换律
			\item 存在零元
			\item 存在加法逆元(负元)
			\item 乘法结合律
			\item 乘法对加法的左分配律和右分配律
		\end{enumerate}
	\end{defi}
	\begin{defi}[域]
		有单位元且非零元可逆的交换环.
	\end{defi}
\end{frame}

\subsection{线性空间的概念}
\frame{\subsectionpage}

\begin{frame}{线性空间}
	\begin{defi}[加法和数乘]
	\end{defi}
	\begin{defi}[线性空间]
		定义了加法和数乘的集合称为线性空间如果其运算满足
		\begin{enumerate}
			\item 加法交换律
			\item 加法和乘法的结合律
			\item 加法零元存在
			\item 加法逆元存在
			\item 域的单位元是数乘单位元
			\item 左分配律和右分配律
		\end{enumerate}
	\end{defi}
\end{frame}

\begin{frame}{向量\quad 线性空间举例}
	\begin{defi}[向量]
		线性空间中的元素称为向量.
	\end{defi}
	\begin{exmp}[什么是线性空间]
		\begin{itemize}
			\item $\mathbb{F}^n$对于我们之前定义的加法和数乘.
			\item $\mathbb{R}^3$对于我们高中学过的矢量的加法和数乘(实际上是上一条的特例, 但这是最直观的例子, 所以单独列出).\footnote{当我们说一个集合是线性空间时, 我们必须指出加法和乘法运算分别是什么, 但由于我实在码不动字了, 在后面几个例子中, 对于平凡的加法和乘法不再做特殊说明.}
			\item 定义某个区间某个区间上的可导函数.
			\item 定义在某个区间上的黎曼可积函数.
		\end{itemize}
	\end{exmp}
\end{frame}

\begin{frame}{线性空间举例}
	\begin{exmp}[什么是线性空间.cont]
		\begin{itemize}
			\item 域上的多项式.
			\item 齐次线性方程组的解.
			\item 线性微分方程的解.
			\item 量子力学中的态空间.
		\end{itemize}
	\end{exmp}
	\begin{exmp}[什么不是线性空间]
		\begin{itemize}
			\item 起点在原点终点在一球面上的矢量, 对于矢量的加法和数乘.
			\item 非齐次方程的解.
		\end{itemize}
	\end{exmp}
\end{frame}

\begin{frame}{研究线性空间的几个途径}
\begin{columns}
	\column{.5\textwidth}
	\begin{itemize}
		\item 从元素的角度
		\item 从子集的角度
		\item 从集合划分的角度
		\item 从线性空间之间关系的角度
	\end{itemize}
	\column{.5\textwidth}
	\begin{itemize}
		\item 基与维数
		\item 子空间与子空间的直和
		\item 等价类, 商集和商空间
		\item 众多线性空间之相同的结构相同的结构
	\end{itemize}
\end{columns}
\end{frame}

\subsection{线性空间的基与维数}
\frame{\subsectionpage}

\begin{frame}{向量组}
	\begin{defi}
		向量组\quad 线性组合\quad 线性表出\quad 有限维线性空间\quad 线性无关(线性独立)\quad 线性相关\quad 极大线性无关组
	\end{defi}
	\begin{prop}
		向量组$A$线性表出线性无关的$B$则$A$中向量个数大于等于$B$.\\
		向量组的不同极大线性无关组所含向量个数相等.
	\end{prop}
	\begin{rmk}
		我们目前仅讨论有限维线性空间.\footnote{遗憾的是, 物理中我们要和大量的无穷维线性空间打交道. 大部分学物理的学生通过将这些概念自然推广来解决这一问题. 虽然这种推广有时在数学上需要一些更艰深的技巧, 幸运的是, 结果常常是对的.}
	\end{rmk}
\end{frame}

\begin{frame}{基与维数}
	\begin{defi}
		基\quad 维数
	\end{defi}
	\begin{prop}
		展开的唯一性.\\
		基的存在性.\footnote{注意我们已经强调我们仅讨论有限维线性空间. 这一定理对无穷维线性空间也是成立的, 其证明需要用到佐恩引理或类似的集合论中的基本定理(公理).}\\
		基中所含向量个数个数的唯一性.(由此可以定义维数)\\
		维数与线性空间中线性无关向量组的规模.($n\leq\dim V$)
	\end{prop}
	\begin{rmk}
		从展开的唯一性看"线性独立"的意义.
	\end{rmk}
\end{frame}

\begin{frame}{基变换和坐标变换}
	\begin{thrm}[基变换与坐标变换]
		设$A$是$n\times n$的可逆矩阵, $\left(\mathbf{e}_1,\cdots,\mathbf{e}_n\right)$和$\left(\mathbf{e}_1',\cdots,\mathbf{e}_n'\right)$是两组基且满足
		\begin{equation}
		\left(\mathbf{e}_1',\cdots,\mathbf{e}_n'\right)=\left(\mathbf{e}_1,\cdots,\mathbf{e}_n\right)A,
		\end{equation}
		矢量$\mathbf{x}=\left(\mathbf{e}_1,\cdots,\mathbf{e}_n\right)x=\left(\mathbf{e}_1',\cdots,\mathbf{e}_n'\right)x'$, 其中$x,x'$为坐标, 是列向量, 则有坐标变换\footnote{这两种变换形式, 一种称为"协变", 一种称为"逆变". 结合对偶空间中的变换, 我们将会发现这两种变换形式有很深刻的内容.}
		\begin{equation}
		x=Ax'.
		\end{equation}
	\end{thrm}
\end{frame}

\begin{frame}{例题}
	\begin{exmp}
		证明$\mathbb{R}$上的$n$级对称矩阵构成线性空间, 并求出它的维数.
	\end{exmp}
\end{frame}

\begin{frame}{例题}
\begin{exmp}
	在定义域为实数集$\mathbb{R}$的所有实值函数形成的线性空间$\mathbb{R}^\mathbb{R}$中
	\begin{enumerate}
		\item 题干的表述有什么问题?
		\item $\sin x, \cos x, \e^x\sin x$是否线性无关?
		\item 对其中$n$个$n-1$阶连续可导函数$f_1(x),\cdots,f_n(x)$定义朗斯基行列式为
		\begin{equation*}
		W(x)=\left|
		\begin{array}{cccc}
		f_1(x)&f_2(x)&\cdots&f_n(x)\\
		f_1'(x)&f_2'(x)&\cdots&f_n'(x)\\
		\vdots&\vdots&&\vdots\\
		f_1^{(n-1)}(x)&f_2^{(n-1)}(x)&\cdots&f_n^{(n-1)}(x)\\
		\end{array}\right|.
		\end{equation*}
		证明若存在$x_0\in\mathbb{R}$使得$W(x_0)\neq0$, 则这些函数线性无关.
	\end{enumerate}
\end{exmp}
\end{frame}

\subsection{子空间与子空间的直和}
\frame{\subsectionpage}

\begin{frame}{子空间\quad 子空间的和}
	\begin{defi}
		子空间\quad 子空间的和\quad
	\end{defi}
	\begin{prop}
		\begin{itemize}
			\item 加法和数乘封闭的子集为子空间.
			\item 两个子空间的交仍是子空间.
			\item 两个子空间的和仍是子空间, 维数为$\dim (V_1+V_2)=\dim V_1+\dim V_2-\dim(V_1\bigcap V_2)并$.
		\end{itemize}
	\end{prop}
\end{frame}

\begin{frame}{直和}
\begin{defi}
	子空间的直和.(分解的唯一性)\quad 补空间
\end{defi}
\begin{prop}[直和的等价表述]
	\begin{enumerate}
		\item $V_1+V_2$是直和.
		\item $V_1+V_2$中零向量的表示唯一.
		\item $V_1\bigcap V_2=0$.
		\item $\dim (V_1+V_2)=\dim V_1+\dim V_2$.
		\item $V_1$的一个基与$V_2$的一个基合起来是$V_1+V_2$的一个基. 
	\end{enumerate}
\end{prop}
\begin{prop}
	补空间存在.\footnote{有限维的证明是容易的.}
\end{prop}
\end{frame}

\begin{frame}{直和}
\begin{prop}[多个子空间直和的等价表述]
	\begin{enumerate}
		\item $V_1+\cdots+V_s$是直和.
		\item $V_1+\cdots+V_s$中零向量的表示唯一.
		\item $V_i\bigcap \left(\sum_{j\neq i}V_j\right)=0$.
		\item $\dim (V_1+\cdots+V_s)=\dim V_1+\cdots+\dim V_s$.
		\item $V_1$的一个基, $V_2$的一个基, $\cdots$, $V_s$的一个基合起来是$V_1+\cdots+V_s$的一个基. 
	\end{enumerate}
\end{prop}
\end{frame}

\subsection{线性空间的同构}
\frame{\subsectionpage}

\begin{frame}{线性空的的同构}
\begin{defi}
	同构映射(保持加法和数乘的双射)\quad 同构
\end{defi}
\begin{prop}
	同构保持了以下关系:
	\begin{itemize}
		\item 零元
		\item 负元
		\item 线性表出和线性相关性
		\item 基
		\item 维数
		\item 子空间
	\end{itemize}
\end{prop}
\end{frame}

\begin{frame}{有限维线性空间的结构}
\begin{thrm}
	$\mathbb{F}$上两个有限维线性空间同构的充分必要条件是它们的维数相等.
\end{thrm}
\begin{rmk}
	$\mathbb{F}$上所有$n$维线性空间都与$\mathbb{F}^n$同构.
\end{rmk}
\end{frame}

\begin{frame}{例题}
	\begin{exmp}
		设集合$X=\{x_1,\cdots, x_n\}$, 求$X$到$\mathbb{F}$上所有映射构成的$\mathbb{F}$上线性空间$\mathbb{F}^X$的一个基和维数. 并写出$f\in\mathbb{F}^X$在这组基下的坐标.
	\end{exmp}
\end{frame}

\subsection{商空间}
\frame{\subsectionpage}

\begin{frame}{等价关系\quad 等价类\quad 商集}
	\begin{defi}[等价关系$\thicksim$]
		如果一个非空集合$S$的一个二元关系$R$满足
		\begin{enumerate}
			\item 反身性: $aRa, \forall a\in S$
			\item 对称性: $aRb\Rightarrow bRa$
			\item 传递性: $aRb, bRc\Rightarrow aRc$
		\end{enumerate}
		则称$R$是一个等价关系.\footnote{等价关系通常记作$\thicksim$, 所以我在标题中写这个符号并不是为了卖萌.}
	\end{defi}
	\begin{defi}[划分\quad 等价类]
	\end{defi}
	\begin{defi}[商集$S/\thicksim$]
		等价类作元素构成的集合.
	\end{defi}
\end{frame}

\begin{frame}{商空间}
	\begin{defi}
		设$U$是$V$的子空间, 定义$V$上的等价关系$\alpha\thicksim\beta: \alpha-\beta\in U$, 我们将$\alpha$所在的等价类记为$\alpha+U$, 称其为$W$的一个陪集. 定义陪集的加法和数乘:
		\begin{itemize}
			\item $(\alpha+W)+(\beta+W):=(\alpha+\beta)+W$
			\item $k(\alpha+W):=k\alpha+W$
		\end{itemize}
	\end{defi}
	\begin{defi}[商空间]
		上面定义的等价关系定出线性空间$V$的一个商集, 这个商集对于上面定义的加法和数乘构成线性空间, 称为商空间. 记作$V/U$.
	\end{defi}
\end{frame}

\begin{frame}{商空间}
\begin{rmk}
	商空间$V/U$中的向量(元素)是$V$的子集(等价类), 而不是$V$中的向量.
\end{rmk}
\begin{prop}
	商空间的维数.\\
	商空间与补空间的同构.
\end{prop}
\begin{exmp}
	非齐次线性方程组的解集是商空间的元素.
\end{exmp}
\end{frame}

\begin{frame}{例题}
	证明上一页写出的性质.
\end{frame}

\end{document}
